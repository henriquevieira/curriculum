%%%%%%%%%%%%%%%%%%%%%%%%%%%%%%%%%%%%%%%%%
% Twenty Seconds Resume/CV
% LaTeX Template
% Version 1.0 (14/7/16)
%
% Original author:
% Carmine Spagnuolo (cspagnuolo@unisa.it) with major modifications by 
% Vel (vel@LaTeXTemplates.com) and Harsh (harsh.gadgil@gmail.com)
%
% License:
% The MIT License (see included LICENSE file)
%
%%%%%%%%%%%%%%%%%%%%%%%%%%%%%%%%%%%%%%%%%

%----------------------------------------------------------------------------------------
%	PACKAGES AND OTHER DOCUMENT CONFIGURATIONS
%----------------------------------------------------------------------------------------

\documentclass[letterpaper]{twentysecondcv} % a4paper for A4

% Command for printing skill overview bubbles
\newcommand\skills{ 
~
	\smartdiagram[bubble diagram]{
        \textbf{Desenvolvedor},
        \textbf{Ciência de dados},
        \textbf{Metodologia}\\\textbf{Ágil},
        \textbf{Aprendizado}\\\textbf{de máquina},
        \textbf{~~Automação~~}\\\textbf{de testes},
        \textbf{Análise}\\\textbf{Estatística}
    }
}

% Programming skill bars
\developer{
{SQL $\textbullet$ XP $\textbullet$ Scrum / 3},
{Shiny $\textbullet$ Django $\textbullet$ Hibernate / 4 },
{Shell script $\textbullet$ Python $\textbullet$ JS / 4},
{R $\textbullet$ Perl $\textbullet$ Java / 5.5}}

\rs{
{Delineamento de experimentos / 4},
{Aprendizado de Máquina / 5.5},
{Design de experimento e análise / 5},
{Análise descritiva / 5.5 }}



%----------------------------------------------------------------------------------------
%	 PERSONAL INFORMATION
%----------------------------------------------------------------------------------------
% If you don't need one or more of the below, just remove the content leaving the command, e.g. \cvnumberphone{}

\cvname{Henrique\\Cursino Vieira} % Your name
\cvjobtitle{ Cientista de dados } % Job
% title/career

\cvlinkedin{/in/henrique-cursino-vieira}
\cvgithub{github.com/henriquevieira}
\cvnumberphone{+55 11 96920 3633} % Phone number
\cvsite{http:vision.ime.usp.br/\~{}hen-rivieira} % Personal website
\cvmail{enrikevieira@gmail.com} % Email address

%----------------------------------------------------------------------------------------

\begin{document}

\makeprofile % Print the sidebar

\section{Perfil profissional}

Sou dedicado, atencioso e estou à disposição para novas funções e desafios. Tenho bom relacionamento interpessoal, dinâmica, flexibilidade. Procuro estar sempre atualizado para meu aprimoramento pessoal e profissional. Gosto de resolver desafios de qualquer natureza, prezo pela busca de novas respostas e novos pontos de vista. \vspace{2mm}

%----------------------------------------------------------------------------------------
%	 EDUCATION
%----------------------------------------------------------------------------------------
\section{Educação}

\begin{twenty} % Environment for a list with descriptions
	\twentyitem
    	{2012 - 2016}
        {}
        {MSc., Bioinformática \textnormal}
        {\href{http://www5.usp.br/}{Universidade de São Paulo, Brasil}}
        {}
        {}
	\twentyitem
    	{2007 - 2010}
		{}
        {BS., Sistemas de Informação \textnormal}
        {\href{http://www.unipune.ac.in/}{Universidade São Judas Tadeu, Brasil}}
        {}
        {}
	%\twentyitem{<dates>}{<title>}{<organization>}{<location>}{<description>}
\end{twenty}

%----------------------------------------------------------------------------------------
%	 EXPERIENCE
%----------------------------------------------------------------------------------------

\section{Experiência}

\begin{twenty} % Environment for a list with descriptions
\twentyitem
    	{Nov 2016 - }
		{Present}
        {Técnico em Bioinformática - Bolsa Técnica 4}
        {\href{http://cetics.butantan.gov.br/}{LETA - CeTICS}}
        {}
        {\begin{itemize}
        \item Desenvolvimento de ferramentas para análise de dados de trans-criptômica e proteômica para o CeTICSdb, software para armazenamento, gerenciamento e analise de dados de transcriptômica e proteômica, desenvolvido pela equipe de bioinformática do LETA - CeTICS - Instituto Butantan.
        \end{itemize}}
        \\
	\twentyitem
    	{Jan 2015 - }
		{Nov 2016}
        {Analista de sistemas Jr.}
        {\href{www.looplex.com.br/}{Looplex}}
        {}
        {
        {\begin{itemize}
        \item Desenvolvimento do software de automação de geração de contratos da startup lawtech Looplex. Aplicação da metodologia de desenvolvimento SCRUM e automação de testes para desenvolvimento ágil.
    \end{itemize}}
        }
    \\   
    \twentyitem
   		{Out 2012 - }
		{Dec 2014}
        {Mestrado com Bolsa CAPES }
        {\href{www.ipqhc.org.br}{ Inst. de Psiquiatria do Hospital das Clínicas de São Paulo}}
        {}
        {
        {\begin{itemize}
        \item Desenvolvimento de pipeline e analises de dados de metilação em DNA com lâminas de microarray.
    \end{itemize}}
        }       
	%\twentyitem{<dates>}{<title>}{<location>}{<description>}
\end{twenty}

\section{Cursos}
\begin{itemize}
	\item Machine Learning - Stanford University, Coursera (Cursando)
	\item Data Scientist with R, Datacamp (Cursando)
	\item Nanodegree Fundamentos de Data Science II, Udacity (Jun. 2018)
	\item Nanodegree Fundamentos de Data Science I, Udacity (Fev. 2018)
	\item R Programming Track, Datacamp (10 cursos, 40h, Jan. 2018)
	\item Basic statistic - University of Amsterdam, Coursera (Mar. 2018)
	\item Advanced School in Big Data Analysis, CeMEAI - USP (48h, Jul. 2017)
	\item SHIFT Big Data Science - Machine Learning and Data Mining, FIAP (32h, Jun. 2016)
	\item Scientific Writing Course, USP (8h, Sep. 2013)
	\item Curso de línguas - Inglês, ADUS (Cursando).
	\item Curso de línguas - Inglês, CNA (Dez. 2011).
	\item Curso de línguas - Japonês, Kumon (Dez. 2009).
\end{itemize}

\section{Pretensão salarial}
\begin{itemize}
	\item 4000,00 a 6000,00
\end{itemize}

\end{document} 
